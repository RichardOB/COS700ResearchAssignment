
%%%%%%%%%%%%%%%%%%%%%%% file typeinst.tex %%%%%%%%%%%%%%%%%%%%%%%%%
%
% This is the LaTeX source for the instructions to authors using
% the LaTeX document class 'llncs.cls' for contributions to
% the Lecture Notes in Computer Sciences series.
% http://www.springer.com/lncs       Springer Heidelberg 2006/05/04
%
% It may be used as a template for your own input - copy it
% to a new file with a new name and use it as the basis
% for your article.
%
% NB: the document class 'llncs' has its own and detailed documentation, see
% ftp://ftp.springer.de/data/pubftp/pub/tex/latex/llncs/latex2e/llncsdoc.pdf
%
%%%%%%%%%%%%%%%%%%%%%%%%%%%%%%%%%%%%%%%%%%%%%%%%%%%%%%%%%%%%%%%%%%%


\documentclass[runningheads,a4paper]{llncs}

\usepackage{amssymb}
\setcounter{tocdepth}{3}
\usepackage{graphicx}

\usepackage{url}
\urldef{\mailsa}\path|richard.obrien.sa@gmail.com|  

\usepackage{cite}
   

\newcommand{\keywords}[1]{\par\addvspace\baselineskip
\noindent\keywordname\enspace\ignorespaces#1}


\begin{document}

\mainmatter  % start of an individual contribution

% first the title is needed
\title{Predicting Gradient Descent Failure of Feed Forward Neural Networks for Classification}



% the name(s) of the author(s) follow(s) next
%
% NB: Chinese authors should write their first names(s) in front of
% their surnames. This ensures that the names appear correctly in
% the running heads and the author index.
%
\author{Richard O'Brien \\ 10688607 \\
Dr. Katerine Malan\and Anna Rakitianskaia}


% the affiliations are given next; don't give your e-mail address
% unless you accept that it will be published
\institute{University of Pretoria, Computer Science Department,\\
COS 700 Research Proposal\\
\mailsa\\
\mailsb\\
\mailsc\\
\url{http://www.cs.up.ac.za}}

%
% NB: a more complex sample for affiliations and the mapping to the
% corresponding authors can be found in the file "llncs.dem"
% (search for the string "\mainmatter" where a contribution starts).
% "llncs.dem" accompanies the document class "llncs.cls".
%

\date{March 2015}
\newpage

\title{Predicting Gradient Descent Failure of Feed Forward Neural Networks for Classification}
% a short form should be given in case it is too long for the running head
\titlerunning{Predicting Gradient Descent Failue of Feed Forward Neural Networks}
\author{}
\institute{}

\authorrunning{10686607}

\maketitle

\begin{abstract}
The abstract should summarize the contents of the paper and should
contain at least 70 and at most 150 words. It should be written using the
\emph{abstract} environment. Content includes 1) a brief background to the research (no citations) and
a problem statement, 2) Method(s) used, 3) Expected Results. This is usually easier to write last and
should not be just a cut and paste from the proposal - write it in a more condensed form.

\keywords{Performance Prediction, Neural Network, Gradient Descent, Error Landscape Analysis}
\end{abstract}


\section{Introduction}

\begin{enumerate}
  \item Introduce the background against which your research will be conducted.
  \item Aim is to convince the reader
  \begin{enumerate}
   \item That the research is relevant, necessary and timely.
   \item That the research has not been done.
  \end{enumerate}
  \item What to put in the introduction:
  \begin{enumerate}
   \item Start with broad field into which your research falls
   \item Become more specific
   \item Lead into the problem statement
   \item Use references throughout (But it is not the full literature survey).
  \end{enumerate}
  \item Example Project: Multicasting of Multimedia via a metropolitan area Network
  \begin{enumerate}
   \item Could start with a statement about how the use of multimedia data has increased
   \item Then move onto the high bandwidth requirements of multimedia data and some ways multimedia data is communicated
   \item Outline the shortcomings of existing approaches
   \item Introduce multicasting and metropolitan area networks
   \item Summarize some of the relevant existing work
  \end{enumerate}
\end{enumerate}


\section{Problem Statement}

\begin{enumerate}
 \item Describe the problem that your research will address in as much detail as possible.
 \item Stating your title as a question sometime helps in naming the problem.
 \item Specify and describe your main objectives and sub-objectives.
 \begin{enumerate}
  \item to demonstrate ...
  \item to evaluate ...
  \item todetermine ...
  \item to establish ...
  \item to argue ...
  \item to prove ...
  \item etc
 \end{enumerate}
  \item Define the scope and limitations of the research
  \begin{enumerate}
   \item What your work deals with and what it does not deal with
   \item ``This work does not consider...''
  \end{enumerate}
  \item State any assumptions that you are aware of.
\end{enumerate}

\section{Methodology}

\begin{enumerate}
 \item Identify the methods that you plan to use to address your objectives. \footnote{Refer back to Lecture 3 handout}
 \begin{enumerate}
  \item Which paradigms and tools?
  \item What code do you expect to write?
  \item What experiments will you need to conduct? (Briefly outline)
 \end{enumerate}
 \item Must tie back to your research objectives.
 \item Should be clear that methods are appropriate \footnote{Explain why alternative methods are not proposed}

\end{enumerate}


\section{Literature Survey}

Gradient Descent, a popular optimisation algorithm, forms the basis of Error Backpropagation which is 
the most widely used training algorithm for Multilayer Feedforward Neural Networks (FFNNs). 
When solving classification and function approximation problems, FFNNs are the preferred neural network architectures 
due to their learning and generalization abilities \cite{gong2012training}. Gradient Descent is even used in oil and gas pipelines, 
training Traditional BP Neural Networks which are responsible for the condition recognition of pipelines \cite{laibin2009novel}.

Despite it's popularity, there exists complexities and limitations intrinsic to Gradient Descent. Gradient Descent is known to fail in
several different ways, including but not limited to the algorithm reaching a poor local minima, reaching an unsatisfactory glabal minima,
encountering numerical precision problems, experiencing slow convergence time, and getting stuck in long plateaus \cite{baldi1995gradient,cetin1993global,laibin2009novel,soni2013performance}

However, the reasons for the above limitations are not independant or well understoond \cite{baldi1995gradient}. Knowing in advance, where Gradient 
Descent is going to fail will help to avoid uneccessary function costs. In order to gain a deeper understanding of what situations Gradient Descent is likely to fail
will require an anlaysis of low level problem characteristics.

\begin{enumerate}
 \item Give an overview of research that has been conducted on the problem you are investigating.\footnote{See Lecture 2}
 \item Only use published references
 \item Work should be directly linked to your work. \footnote{You must point out this link}
 \item Objective: To show how your research relates to and differs from existing research.
 \item Identify the main players in the field.
\end{enumerate}


\section{Project Planning}

\begin{enumerate}

 \item Identify and describe the specific phases of your work
 \begin{enumerate}
  \item Such as reading, writing, coding, experimentation, testing, writing, editing, writing.
 \end{enumerate}
 
 \item Provide a schedule of everything that needs to be done all the way up to the submission date (5 November) or Presentation Date
 \begin{enumerate}
  \item Tasks that need to be done and the time allocated to them with deadlines.
  \item Remember that your supervisor needs time to review your report/paper and you need time to incorporate changes before the submission date.
 \end{enumerate}
 
 \item Advantages of Task Timelines:
 \begin{enumerate}
  \item Forces you to break down the work into manageable chunks.
  \item Lets you think about the logical sequence of tasks.
  \item Planning helps you to not miss out important tasks.
  \item Creates deadlines for yourself.
  \item Clears your head
  \item Can be used to map progress.
 \end{enumerate}
 
 \item Some Hints:
 \begin{enumerate}
  \item Be realistic, rather pad the time than set unrealistic deadlines.
  \item Don't forget to factor in other things (other modules, your life, ...)
  \item Some tasks can run concurrently - good to have alternative tasks to work on.
  \item Give page estimates to writing tasks.
  \item Start from end and plan backwards.
  \item Commit to your plan: have your timeline visible somewhere
  \item Update your plan if necessary as you progress through the work.
 \end{enumerate}

\end{enumerate}

\section{BibTeX Entries}

The correct BibTeX entries for the Lecture Notes in Computer Science
volumes can be found at the following Website shortly after the
publication of the book:
\url{http://www.informatik.uni-trier.de/~ley/db/journals/lncs.html}

\subsubsection*{Acknowledgments.} The heading should be treated as a
subsubsection heading and should not be assigned a number.

\section{References}\label{references}

Provide a list of references that you have cited in the proposal. See Lecture 2.\\

This is a cite example \cite{mersmann2011exploratory}

\bibliographystyle{plain}
\bibliography{bibfile}

\end{document}
